% Created 2018-07-22 日 23:00
% Intended LaTeX compiler: pdflatex
\documentclass[11pt]{article}
\usepackage[utf8]{inputenc}
\usepackage[T1]{fontenc}
\usepackage{graphicx}
\usepackage{grffile}
\usepackage{longtable}
\usepackage{wrapfig}
\usepackage{rotating}
\usepackage[normalem]{ulem}
\usepackage{amsmath}
\usepackage{textcomp}
\usepackage{amssymb}
\usepackage{capt-of}
\usepackage{hyperref}
\author{conao}
\date{\textit{<2018-07-17 火 09:30>}}
\title{Emacsを使う最大の理由!文書作成のすべてをorg-modeで記述しよう!}
\hypersetup{
 pdfauthor={conao},
 pdftitle={Emacsを使う最大の理由!文書作成のすべてをorg-modeで記述しよう!},
 pdfkeywords={},
 pdfsubject={},
 pdfcreator={Emacs 25.3.1 (Org mode 9.1.13)}, 
 pdflang={English}}
\begin{document}

\maketitle
\tableofcontents

Emacsを使う理由は様々あると思いますが、私がEmacsを使う最大の理由は \textbf{org-modeが使えること} です。
会議の議事録や、規約の編集、卒業論文まで全ての文書作成をorg-modeで行うことができます。

\section{環境}
\label{sec:org71cfa91}
\begin{itemize}
\item \href{../emacs/index.org}{Emacs - 25.3.1}
\begin{itemize}
\item \href{../use-package.org}{use-package - 2.4}
\item org-mode - 9.1.13
\end{itemize}
\end{itemize}

\section{はじめに}
\label{sec:org75bfb36}
org-modeで記述する利点としては、プレーンテキストで文書を作成するため \textbf{diffがみやすくなる} という点があります。
最近ではWardのようなリッチ文書作成ソフトでもdiffを取りやすくなっているようですが、プレーンテキストにはかないません。
git管理することで、その文書を \textbf{誰が}, \textbf{いつ} 書き換えたのか明白に管理できますし、巻き戻しも容易です。
コミットメッセージを活用することで \textbf{なぜ} 書き換えたのかも管理できます。

そうした「管理された文書」を思い通りの形式に \textbf{変換して出力} できます。
例に挙げた会議の議事録や、規約の編集は所属していた交響楽団に関するものでした。
もちろん団員はEmacsをインストールしていないため、純粋なプレーンテキストとしてのorg-modeは理解しづらいですが、
\textbf{HTMLに出力して印刷し配布} しましたし、後輩には \textbf{Word形式} で引き継ぐことができました。

卒業論文はもちろん \textbf{\LaTeX{}} での提出が求められます。この要求に対してもorg-modeは十二分に応えてくれます。
なぜならorg-modeの文法は簡潔で \textbf{執筆に集中できる} 上に、 \textbf{編集のためのお助け機能} をorg-modeが提供してくれているからです。

ここだけの話ですが、HTMLや\LaTeX{}は \textbf{人間が書くものじゃない} と個人的に考えています。
HTMLはZen Codingなどのお助け機能や\LaTeX{}もEmacsではYaTeXなどのお助け機能がありますが、根本的に \textbf{人間が書くフォーマットじゃない} んです。

その点org-modeを使えば執筆に集中でき、難しい構造を隠してくれる上に、 \textbf{多彩なフォーマットに出力} できるのです。
もう人類はHTMLタグや\LaTeX{}環境、通常マークダウンとQiita記法やはてな記法の違いなどに頭を悩ますことはないのです。
なぜならorg-mode記法で一回書けば全て変換できるからです!

ちょっと話が大きくなりましたが、本心です。そして私がorg-modeに出会ったときの衝撃です。
ちなみにEmacsは気難しいやつで当時何度も挫折していた頃でしたが、 \textbf{org-modeを使うため} にEmacsと仲良くなって
今ではEmacsで全てのプログラミング、文書作成を行っています笑
\section{インストール}
\label{sec:org2563797}
\begin{verbatim}
(require 'package)
(when (not (package-installed-p 'use-package))
  (package-refresh-contents)
  (package-install 'use-package))
(use-package org :ensure t :defer t
  :bind (("C-c o l" . org-store-link)
	 ("C-c o a" . org-agenda)
	 ("C-c o c" . org-capture)
	 ("C-c o b" . org-iswitchb))
  :config
  (setq org-src-window-setup     'other-window
	org-use-sub-superscripts '{}))
\end{verbatim}
\captionof{figure}{\label{orgf70da73}
org-modeのインストール}
上記はgithubで\href{https://github.com/conao/dotfiles}{公開}してる.emacs.dからの抜粋です。
\texttt{"use-package org"} と検索して実際のコードを見てみるといいかも知れません。
\section{org-modeカテゴリ}
\label{sec:orge34015c}
org-modeは巨大で、関連するパッケージも多いため小記事に分けます。
\begin{itemize}
\item \href{./basis-syntax.org}{org-mode基本文法}
\end{itemize}

\(\sqrt{2}\)
\end{document}